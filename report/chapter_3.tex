\chapter{Пользовательский интерфейс}
В курсовой работе реализован графический пользовательский веб-интерфейс, доступный например через веб-браузер.

Для программной реализации был использован язык программирования: \smallcode{TypeScript} и библиотека \smallcode{React}. Для сборки \smallcode{js-bundle} использовалась утилита \smallcode{eslint}.

\section{Генерация \smallcode{SELECT}-запросов}
Генерация \smallcode{SELECT}-запросов выполняется на уровне клиента (пользователя). Для данной работы выбрано следующее подмножество \smallcode{SELECT}-запросов:
\begin{enumerate}
	\item Выбор одной или нескольких колонок.
	\item Агрегатные функции для колонок.
	\item Единственное условие \smallcode{WHERE} (опционально).
\end{enumerate}
Также для генератора запроса заданы следующие ограничения:
\begin{itemize}
	\item наполнение данных в форме при генерации синхронизируется если выбрать таблицу либо колонку.
	\item сгенерированные запросы сохраняются в приложении.
\end{itemize}

\subsection*{Сохранение \smallcode{SELECT}-запросов}
Для реализации сохранения сгенерированных \smallcode{SELECT}-запросов была введена дополнительная таблица в базу данных метаданных \smallcode{history}. При выполнении запроса всегда фиксируем историю всех запросов в базе данных для воспроизведения их в дальнейшем.

На \hyperref[fig:schema-image2]{Рисунке \ref*{fig:schema-image2}} представлена обновленная реляционная схема базы данных метаданных, хранящая историю запросов.\\

\newpage
\includepdf[
scale=.89,
pages=-,
pagecommand={%
	\thispagestyle{empty}% или empty, если без номера страницы
	\addtocounter{figure}{1}%
	\refstepcounter{figure}%
	\vspace*{10cm}%
	\hspace{22.5cm}
	\begin{turn}{90}%
		\centering
		{Рисунок~\thefigure\ --- Схема базы данных с учетом таблицы \smallcode{history}}\label{fig:schema-image2}%
	\end{turn}%
	\begin{textblock*}{5cm}(275mm,13cm) % (x, y) в cm от верхнего левого угла страницы
		\begin{turn}{90}
			\thepage
		\end{turn}
	\end{textblock*}
},
nup=1x1,
landscape=true,
fitpaper=true,
templatesize={420mm}{297mm},
offset=0mm 10mm
]{schema-image2.jpg}
\addtocounter{page}{0}
\newpage
С точки зрения абстракции не совсем правильно хранить в базе данных метаданных запросы, однако для более удобной разработки в курсовой работе данные сущности располагаются рядом.

\section{Особенности пользовательского интерфейса}

В пользовательском интерфейсе определены следующие основные части:
\begin{enumerate}
	\item Список подключенных баз данных.
	\item Форма добавления новой базы данных по \smallcode{DSN}.
	\item Генератор \smallcode{SELECT}-запроса.
	\item История сгенерированных \smallcode{SELECT}-запросов.
	\item Модальная форма результата \smallcode{SELECT}-запроса.
\end{enumerate}

\subsection*{Список подключенных баз данных}
На \hyperref[fig:ui1]{Рисунке \ref*{fig:ui1}} представлен внешний вид компонентов пользовательского интерфейса: 
\begin{itemize}[]
	\item список подключенных баз данных;
	\item форма добавления новой базы данных;
\end{itemize}
\begin{figure}[h!]
	\centering
	\includegraphics[width=.6\linewidth]{ui1.jpg}
	\caption{Список подключений и форма создания нового подключения}
	\label{fig:ui1}
\end{figure}
\newpage
\subsection*{Генератор \smallcode{SELECT}-запроса}
На \hyperref[fig:ui2]{Рисунке \ref*{fig:ui2}} представлен внешний вид генератора \smallcode{SELECT}-запроса.
\begin{figure}[h!]
	\centering
	\includegraphics[width=.8\linewidth]{ui2.jpg}
	\caption{Генератор \smallcode{SELECT}-запроса}
	\label{fig:ui2}
\end{figure}

В генераторе запроса пользователю доступен выбор:
\begin{itemize}[]
	\item одной или нескольких колонок;
	\item опциональные агрегатные функции (\smallcode{COUNT, SUM, MIN, MAX, AVG});
	\item опциональные условия \smallcode{WHERE};
\end{itemize}


Наполнение select-box'ов зависит от выбранных колонок или таблицы и синхронизируется в зависимости выбора пользователя.

\subsection*{История сгенерированных \smallcode{SELECT}-запросов}
На \hyperref[fig:ui3]{Рисунке \ref*{fig:ui3}} представлен внешний вид истории сгенерированных \smallcode{SELECT}-запросов.
\newpage
\begin{figure}[h!]
	\centering
	\includegraphics[width=1\linewidth]{ui3.jpg}
	\caption{История \smallcode{SELECT}-запросов}
	\label{fig:ui3}
\end{figure}

\subsection*{Модальное окно результата \smallcode{SELECT}-запроса}
На \hyperref[fig:ui4]{Рисунке \ref*{fig:ui4}} представлен внешний вид модального окна выполненного \smallcode{SELECT}-запроса.
\begin{figure}[h!]
	\centering
	\includegraphics[width=.7\linewidth]{ui4.jpg}
	\caption{Результат \smallcode{SELECT}-запроса}
	\label{fig:ui4}
\end{figure}

Пользовательский интерфейс позволяет выполнять генерацию \smallcode{SELECT}-запросов и их выполнение для добавленных баз данных. За счет разделения сущностей пользовательский интерфейс обращается через \smallcode{API} к \smallcode{Metadata Manager} веб-серверу и не выполняет напрямую манипуляции с внешними базами данных.

