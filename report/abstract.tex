\chapter*{}
\thispagestyle{empty}

\begin{center}
	\textbf{РЕФЕРАТ}
\end{center}

На 31~с., 14~рисунков, 1~приложение.


КЛЮЧЕВЫЕ СЛОВА: Метаданные, базы данных, PostgreSQL, СУБД, клиент-серверная архитектура, веб-приложение.


Тема курсовой работы: "<Управление метаданными реляционной системы управления базы данных PostgreSQL">

Предметом исследования курсовой работы является проектирование и разработка приложения для управлением метаданными реляционной базы данных PostgreSQL и реализацией \smallcode{SELECT} запросов.

Целью курсовой работы является разработка веб-приложения с пользовательским графическим интерфейсом для управления метаданными.


В работе рассмотрены теоретические аспекты связанные с метаданными (\smallcode{information schema}) системы управления реляционными базами данных PostgreSQL. Выполнено проектирование и разработка веб-приложения с пользовательским графическим интерфейсом. Рассмотрены три удаленных СУБД, для которых выполнялись запросы \smallcode{SELECT}. СУБД были расположены на удаленном ресурсе, однако приложение поддерживает подключение к любой СУБД PostgreSQL при заданном \smallcode{DSN}. Разработанное приложение доступно пользователю из веб-браузера, в котором происходит взаимодействие с СУБД.

\begin{center}
	\textbf{ABSTRACT}
\end{center}
31 pages, 14 figures, 1 appendices


KEYWORDS: Metadata, databases, PostgreSQL, MSDB, client-server architecture, web-application

Title of the thesis: "Metadata management of the PostgreSQL relational Database Management System"

The subject of this coursework is the design and development of an application for managing metadata of a relational PostgreSQL database and executing \smallcode{SELECT} queries.

The purpose of the coursework is to develop a web application with a graphical user interface for metadata management.

The work examines the theoretical aspects related to metadata (\smallcode{information schema}) of the PostgreSQL relational database management system. The design and development of a web application with a graphical user interface were carried out. Three remote DBMS instances were considered, for which \smallcode{SELECT} queries were executed. The DBMS instances were located on a remote resource; however, the application supports connection to any PostgreSQL DBMS given a \smallcode{DSN}. The developed application is available to the user through a web browser, where interaction with the DBMS takes place.

\newpage
\endinput