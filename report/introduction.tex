\chapter*{Введение}
\addcontentsline{toc}{chapter}{Введение}
При работе с уже существующими базами данных возникает вопрос неизвестности ее внутренней структуры и необходимостью пользователю базы данных самостоятельно разрешать конфликты связанные с этим. Внутренняя структура базы данных: таблица, колонки, связи внутри базы данных можно обобщить в сущность метаданных базы данных.


Существуют готовые клиенты для удобной работы с внешними базами данных такие как: \smallcode{pgAdmin}, \smallcode{workBench}, \smallcode{MSAccess}. Данные решения предоставляют административные и пользовательские функции для работы с базами данных.


Целью данной курсовой работы является разработка приложения, предназначенного для работы с метаданными реляционных баз данных \smallcode{PostgreSQL}. Разрабатываемое приложение должно обеспечивать подключение к произвольной базе данных, автоматическое извлечение информации о её структуре, сохранение метаданных во внутреннем хранилище и предоставление графического интерфейса для визуального анализа и построения \smallcode{SQL}-запросов типа \smallcode{SELECT}.


\chapter*{Постановка задачи}
\addcontentsline{toc}{chapter}{Постановка задачи}
В рамках работы необходимо реализовать приложение для работы с метаданными реляционной системы управления базой данных PostgreSQL и реализовать генератор подмножества \texttt{SELECT} запросов для заданных баз данных, а также хранение истории сгенерированных запросов. Для решения данной задачи выделены следующие подзадачи:
\begin{enumerate}
	\item Описание заданного окружения.
	\item Создание хранилища метаданных:
	\begin{itemize}
		\item проектирование схемы базы данных метаданных;
		\item составление запросов к \smallcode{information schema};
		\item реализация сохранения метаданных по DSN-строке подключения к СУБД;
	\end{itemize}
	\item Реализация графического интерфейса.
	\item Разработка генератора подмножества \smallcode{SELECT}.
	\item Организация хранения истории запросов и возможности их повторного выполнения.
\end{enumerate}