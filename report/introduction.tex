\chapter*{Постановка задачи}
\addcontentsline{toc}{chapter}{Постановка задачи}
В рамках работы необходимо реализовать приложение для работы с метаданными реляционной системы управления базой данных PostgreSQL и реализовать генератор подмножества SELECT запросов для заданных баз данных, а также хранение истории сгенерированных запросов. Для решения данной задачи выделены следующие подзадачи:
\begin{enumerate}
	\item Описание заданного окружения.
	\item Создание хранилища метаданных:
	\begin{itemize}
		\item проектирование схемы базы данных метаданных;
		\item составление запросов к \smallcode{pg\_catalog};
		\item реализация сохранения метаданных по DSN-строке подключения к СУБД;
	\end{itemize}
	\item Реализация графического интерфейса.
	\item Разработка генератора подмножества \smallcode{SELECT}.
	\item Организация хранения истории запросов и возможности их повторного выполнения.
\end{enumerate}