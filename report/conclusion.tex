\chapter*{Заключение}
\addcontentsline{toc}{chapter}{Заключение}

В результате курсовой работы разработано приложение для управления работы с метаданными реляционной системы управления базой данных PostgreSQL и реализован генератор подмножества \texttt{SELECT} запросов для заданных баз данных, а также хранение истории сгенерированных запросов. 

Для решения данной задачи были выполнены следующие подзадачи:
\begin{enumerate}
	\item Описание заданного окружения.
	\item Создание хранилища метаданных:
	\begin{itemize}
		\item проектирование схемы базы данных метаданных;
		\item составление запросов к \smallcode{information schema};
		\item реализация сохранения метаданных по DSN-строке подключения к СУБД;
	\end{itemize}
	\item Реализация графического интерфейса.
	\item Разработка генератора подмножества \smallcode{SELECT}.
	\item Организация хранения истории запросов и возможности их повторного выполнения.
\end{enumerate}

Следует отметить, что одним из недостатков рассматриваемой реализации является хранение истории запросов в базе данных метаданных, 
что частично нарушает разделения абстракций. Данный подход был выбран для упрощения реализации на этапе разработки.